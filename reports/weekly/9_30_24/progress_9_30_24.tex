\documentclass{article}
\usepackage{geometry}
\geometry{a4paper, margin=1in}
\usepackage{graphicx}
\usepackage[colorlinks=true, linkcolor=blue, citecolor=blue, urlcolor=blue]{hyperref}
\usepackage{listings}
\usepackage{xcolor}
\usepackage{amsmath}
\usepackage{enumitem}

\lstset{
    language=Python,
    basicstyle=\ttfamily\footnotesize\selectfont, % use the selected monospaced font
    backgroundcolor=\color{white},
    keywordstyle=\color{blue},
    commentstyle=\color{gray},
    stringstyle=\color{red},
    numbers=left,
    numberstyle=\tiny\color{gray},
    stepnumber=1,
    numbersep=10pt,
    frame=single,
    breaklines=true,
    captionpos=b,
    tabsize=4
}

\title{Progress Report - Week 3 \\
\large Novel Methods for Determining Neuron Sensitivity in Large Language Models}
\author{
    [Welby Seely] \\
    \texttt{[wseely@emich.edu]}
}
\date{\today}

\begin{document}

    \maketitle


    \section{Project Overview}\label{sec:project-overview}
    Large Language Model (LLM) power consumption is increasingly problematic for datacenters, with \textit{The Electric Power
    Research Institute} forecasting that data centers may see their electricity consumption double by
    2030~\cite{kindig2024}.

    Optimizing these models is key to reducing power consumption.
    The cornerstone of optimization is determining what you need, and what you don't: determine the importance of
    aspects and components of the system, enabling you to perform systematic optimization with certainty.
    In Artificial Neural Networks (ANN), the atomic unit of these components is the artificial neuron.
    To optimize an ANN, understanding the sensitivity of the neuron to changes in its input and parameters is paramount,
    but the sheer size and complexity of LLMs makes this incredibly challenging.

    Thus, the goal of this research is to:
    \begin{enumerate}
        \item Understand current methodologies for determining neuron sensitivity in LLMs.
        \item Explore improvements that can be made to more accurately and more efficiently determine neuron sensitivity in
        LLMs.
    \end{enumerate}

    The intent is that this research will be a first step towards innovating and improving optimizations for LLMs,
    reducing power consumption and enabling the models to run on lower powered hardware.


    \section{Current Progress}\label{sec:current-progress}

    \subsection{Pruning with LLM--Pruner}\label{subsec:pruning-with-llm--pruner}
    Forked the LLM--Pruner codebase to work on my local machine.
    Optimized perplexity calculation to avoid keeping unnecessary tensor information in VRAM. Was able to get evaluation with GPU and pruning with CPU nominally working, but incorrect metadata of the model is preventing the model from being loaded by external scripts.
    GPU pruning is failing due to out of memory errors - currently working on using $torch.utils.checkpoint$ as a workaround.
    Small changes made so far to the codebase can be found on GitHub ~\cite{llm-pruner-pr}.

    \subsection{Measuring Power Consumption Programmatically}\label{subsec:measuring-power-consumption-locally}
    I've developed a simple test runner for measuring the energy usage of an LLM, using the test dataset of ``wikitext-2-v1'' from HuggingFace~\cite{merity2016pointersentinelmixturemodels}.
    The code can be found on GitHub~\cite{llm-test-runner}.
    Energy consumption is measured using the pynvml library, which is supported on Volta or newer Nvidia GPUs. This will facilitate measuring energy savings on approximated models even when the associated papers do not measure energy savings.

    As an initial proof of concept, only the first 20 rows of the dataset were used.

    \begin{itemize}
        \item Baseline: Llama 3--8B
        \begin{itemize}
            \item Total Execution Time: 87.98 s
            \item Total Energy Usage: 23716 J
            \item Average Time Per Token: 15.31 ms
            \item Average Energy Per Token: 4.12 J
        \end{itemize}
        \item Pruned: Llama--3--Minitron--4B--Width-Base
        \begin{itemize}
            \item Total Execution Time: 146.28 s
            \item Total Energy Usage: 3055.35 J
            \item Average Time Per Token: 15.22 ms
            \item Average Energy Per Token: 3.17 J
        \end{itemize}
        \item Baseline: Llama 2--7B (Decapoda Research)~\cite{decapoda-llama-7B}
        \begin{itemize}
            \item Total Execution Time: 152.31 s
            \item Total Energy Usage: 40887.47 J
            \item Average Time Per Token: 15.23 ms
            \item Average Energy Per Token: 4.89 J
        \end{itemize}
        \item Pruned with LLM--Pruner~\cite{ma2023llm}: Llama 2--7B (Decapoda Research)
        \item TODO - working on getting pruned model to load correctly
    \end{itemize}

    \subsection{Comparative Analysis of Neuron Sensitivity Techniques}\label{subsec:comparative-analysis-of-neuron-sensitivity-techniques}

    As a part of literature review, I've grouped different works into particular categories of importance determination.
    I've also added Nvidia's Minitron~\cite{sreenivas2024llm} to the ``Structure Importance Analysis'' section.

    % Feedback:
% Start actually implementing (or even better try to find open-source implementations) models
% How do these works evaluate their energy savings (if they do)? Do they evaluate actual hardware/simulated hardware?

\subsection{Structure Importance Analysis}\label{subsec:strucure-importance-analysis}

This is an analysis of optimization techniques that rely on understanding on the importance of larger structures with the LLM\@.
\begin{itemize}
    \item Llm-pruner: On the structural pruning of large language models~\cite{ma2023llm}.
    \begin{itemize}
        \item Identifies dependency-based structures, in the context of pruning
        \item Axiom: a neuron $N_j$ is co-dependent on another neuron $N_i$ if and only if it only receives input from $N_j$ and $N_i$ only outputs to $N_j$.
         More formally:
        \begin{equation}
             N_j \in \operatorname{Out}(N_i) \wedge \operatorname{Deg}^-(N_j) = 1 \Rightarrow N_j \text{ is dependent on } N_i\label{eq:equation}
        \end{equation}
        \begin{equation}
             N_i \in \operatorname{In}(N_j) \wedge \operatorname{Deg}^+(N_i) = 1 \Rightarrow N_i \text{ is dependent on } N_j\label{eq:equation2}
        \end{equation}
        \item Identifying groups (building a dependency graph):
        \begin{enumerate}
            \item Select an arbitrary neuron as an initial ``trigger.''
            \item Using the axiom, identify if there is a dependency relationship with another neuron, either forwards or backwards.
            \item Newly identified neurons trigger additional ``dependent'' neurons, if eligible by the axiom.
            \item Continue this process iteratively until no new dependent neurons are found.
        \end{enumerate}
        \item Estimating the importance of a group:
        \begin{enumerate}
            \item Identify a small, publicly available dataset (different from the training data).
            \item Measure the importance of a group by using Vector-Wise and Element Importance in aggregate:
            \begin{enumerate}
                \item Vector-Wise Importance: relate weight matrices to the loss function.
                This is useful because using this non-training data means ${\partial \mathcal{L}}/{\partial W_i} \not \approx 0$.
                \item Element Wise Importance: relate individual parameters to the loss function.
                \item Finally, these two importance scores must be aggregated through a summation, a maximum, a product, or the last parameter in the group.
            \end{enumerate}
        \end{enumerate}
        \item Advantages:
        \begin{itemize}
            \item The simplistic nature of the axiom makes it easy to identify groups eligible for simple pruning.
            \item Element-wise importance calculation is simplified through the use of approximate gradients and Hessians.
            \item The small dataset keeps the calculations to a relative minimum.
        \end{itemize}
        \item These advantages would result in reduced power consumption, but the paper unfortunately does not delve into these details.
        More research into pruning methods is necessary.
        \item Weaknesses:
        \begin{itemize}
            \item The grouping algorithm cannot classify all neurons into groups - ``dependence'' is relegated to a very simplistic definition that prevents us from looking at the entire model from a holistic point of view.
            \item The paper reports a rapid collapse in model accuracy and coherence at $~20\%$ pruning of neurons.
             I suspect that this is a limitation of the axiom, the naive definition of ``dependence'':
             the actual importance of groups of neurons cannot truly be understood when removing the more complex neural relationships from the analysis.
             This is a possible narrowing of focus this research could take.
            \item The grouping algorithm is not foolproof, and fails for certain models and ``operations''\cite{LLM-Pruner}.
        \end{itemize}
    \end{itemize}
\end{itemize}
    % Feedback:
% Start actually implementing (or even better try to find open-source implementations) models
% How do these works evaluate their energy savings (if they do)? Do they evaluate actual hardware/simulated hardware?

\subsubsection{Input Sensitivity Analysis}\label{subsubsec:individual-neuron-importance-analysis}

This is an analysis of optimization techniques with importance measurements based on measuring the sensitivity of a neuron or group of neurons to the DNNs original input features. \@.
\begin{itemize}
    \item Activation Maximization ~\cite{erhan2009visualizing}
    \subitem Optimize an input to maximize the activation of a specific neuron or neuron layer.
    \item Saliency Maps
    \subitem Visualize Neuron sensitivity by highlighting inputs that most affect a particular neuron or output~\cite{hsu2023explainable}.
    \item Layer-wise Relevance Propagation
    \subitem Similar to Saliency Maps, decomposes the prediction of a network back to the individual input features~\cite{jia2022interpreting}.
    \item Integrated Gradients
    \subitem Quantifies feature importance by integrating gradients along the path from a baseline input to the actual input~\cite{sundararajan2017axiomatic}.
    \item SHAP Values (SHapley Additive exPlanation)
    \subitem Quantifies sensitivity by assigning a contribution score to each input feature based on its impact on the model’s output~\cite{nohara2022explanation}.
\end{itemize}
    % Feedback:
% Start actually implementing (or even better try to find open-source implementations) models
% How do these works evaluate their energy savings (if they do)? Do they evaluate actual hardware/simulated hardware?

\subsubsection{Self-Attention Analysis (LLMs)}\label{subsubsec:token-importance-analysis}

This is an analysis of optimization techniques that are more specific to LLMs, relying on understanding the importance of tokens during the evaluation of a LLM\@.

\begin{itemize}
    \item Rethinking the importance analysis in self-attention (2021)l~\cite{shi2021sparsebert}.
    \begin{itemize}
        \item Sensitivity is represented by attention weights associated with an attention head, which dictates the degree of influence of each token in a sequence mode.
        \item Hypothesis: not all positions in the self-attention matrix in a Transformer model are equally important.
        \item Differentiable ArchiTecture Search (DARTS)
        \item Process:
        \begin{enumerate}
            \item Introduce a set of learnable parameters $a_{i,j}$ for each attention position $(i, j)$.
            Each of these parameters is the probability representing the importance of token $i$ to token $j$.
            \item Perform element-wise multiplication with the original attention matrix.
            \item Optimize both the model parameters and attention importance parameters simultaneously during training using gradient descent.
            \item TODO explain specifically how this helps the model and the designer understand degrees of influence of each token.
        \end{enumerate}
    \end{itemize}
\end{itemize}


    \section{Challenges and Issues}\label{sec:challenges-and-issues}
    Even with a 4090 with 24 GB of VRAM, research code is exhausting memory.
    Optimizations are required, and figuring out a way to run backpropagation on Llama 3--8B on GPU without running out of memory is challenging.
    I'm currently exploring $torch.utils.checkpoint$ as a workaround, trading memory cost for computation cost.


    \section{Next Steps}\label{sec:next-steps}
    Will continue working on evaluating LLM models and optimization techniques, evaluating power usage and importance determination techniques.
    Structured analysis of importance is of particular interest.

    \section{Questions or Feedback Needed}\label{sec:questions-or-feedback-needed}
    Based on your feedback, I've begun looking into energy savings and digging into running the models and approximation code.
    \bibliographystyle{plainurl}
    \bibliography{bibliography}
    \thanks Thanks to ChatGPT~\cite{chatgpt_2024} for help with PyTorch and for helping generate BibTeX citations for websites.

\end{document}
