\documentclass{article}
\usepackage{geometry}
\geometry{a4paper, margin=1in}
\usepackage{graphicx}



\title{Progress Report - Week 1 \\
\large Novel Methods for Determining Neuron Sensitivity in Large Language Models}
\author{
    [Welby Seely] \\
    \texttt{[wseely@emich.edu]}
}
\date{\today}

\begin{document}

    \maketitle



    \section*{Project Overview}
    Large Language Model (LLM) power consumption is increasingly problematic for datacenters, with \textit{The Electric Power
    Research Institute} forecasting that data centers may see their electricity consumption double by
    2030~\cite{kindig2024}.

    Optimizing these models is key to reducing power consumption.
    The cornerstone of optimization is determining what you need, and what you don't: determine the importance of
    aspects and components of the system, enabling you to perform systematic optimization with certainty.
    In Artificial Neural Networks (ANN), the atomic unit of these components is the artificial neuron.
    To optimize an ANN, understanding the sensitivity of the neuron to changes in its input and parameters is paramount,
    but the sheer size and complexity of LLMs makes this incredibly challenging.

    Thus, the goal of this research is to:
    \begin{enumerate}
        \item understand current methodologies for determining neuron sensitivity in LLMs
        \item explore improvements that can be made to more accurately and more efficiently determine neuron sensitivity in
        LLMs
    \end{enumerate}

    The intent is that this research will be a first step towards innovating and improving optimizations for LLMs,
    reducing power consumption and enabling the models to run on lower powered hardware.

    \section*{Current Progress}
    \subsection*{Literature Review}
    Initial exploration of the literature regarding neuron sensitivity and component importance has revealed the
    following techniques:

    \begin{itemize}
        \item Coupled Structure Analysis
        \subitem Identifies dependency-based structures, in the context of pruning~\cite{ma2023llm}
        \item Activation Maximization
        \subitem Optimize an input to maximize the activation of a specific neuron or neuron layer
        \item Saliency Maps
        \subitem Visualize Neuron sensitivity by highlighting inputs that most affect a particular neuron or output~\cite{hsu2023explainable}
        \item Layer-wise Relevance Propagation
        \subitem Similar to Saliency Maps, decomposes the prediction of a network back to the individual input features~\cite{jia2022interpreting}
        \item Self-Attention Analysis (LLMs)
        \subitem Sensitivity is represented by attention weights associated with an attention head, which dictates the degree of influence of each token in a sequence model~\cite{shi2021sparsebert}
        \item Integrated Gradients
        \subitem Quantifies feature importance by integrating gradients along the path from a baseline input to the actual input~\cite{sundararajan2017axiomatic}
        \item SHAP Values (SHapley Additive exPlanation)
        \subitem Quantifies sensitivity by assigning a contribution score to each input feature based on its impact on the model’s output~\cite{nohara2022explanation}
        \item Mechanistic Interpretability
        \subitem Maps out neural circuits in order to determine which parts of a model are responsible for specific capabilities (e.g.\ arithmetic, semantic relationships between words)~\cite{singh2024rethinking}
    \end{itemize}

    During this initial review, it became apparent that the literature detailing algorithms for determining component
    ``importance'' in DNNs is general coupled with specific optimization techniques, making understanding the current state
    of determining neuron sensitivity for LLMs difficult.
    A sub-goal of this project is to aggregate these techniques in a single document.

    \subsection*{Tool Familiarization}
    Since writing the proposal, I've begun work on gaining familiarity with TensorFlow\cite{tensorflow_2024} and
    Pytorch\cite{pytorch_2024} by writing a simple ``or'' gate implementation using both frameworks\cite{seely_tensorflow_2024}



    \section*{Challenges and Issues}
    Describe any difficulties or challenges you faced during the past week. What roadblocks have come up? What have you done to resolve them (or what are you planning to do)?

    \section*{Next Steps}
    List the tasks or milestones for the upcoming week. Provide a clear plan for what you intend to work on and how you will tackle any ongoing challenges.

    \section*{Questions or Feedback Needed}
    If there are any areas where you need guidance, feedback, or clarification from me, include them here.

    \bibliographystyle{plainurl}
    \bibliography{bibliography}

\end{document}
